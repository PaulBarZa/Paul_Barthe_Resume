%% Copyright 2006-2010 Xavier Danaux (xdanaux@gmail.com).
%% Copyright 2010-2011 Mark Liu (markwayneliu@gmail.com).
%
% This work may be distributed and/or modified under the
% conditions of the LaTeX Project Public License version 1.3c,
% available at http://www.latex-project.org/lppl/.

\documentclass[10pt,a4paper,sans]{moderncv}

\usepackage{verbatim}
\usepackage{graphicx}
\usepackage{tikz}

% moderncv themes
\moderncvstyle{classic}
\moderncvcolor{blue}

% character encoding
\usepackage[utf8]{inputenc}                   

% adjust the page margins
\usepackage[left=.6in,right=.6in,top=.5in,bottom=.8in]{geometry}

\setlength{\hintscolumnwidth}{3cm}						

% personal data
\firstname{}
\familyname{}
\mobile{07 81 09 14 85}                    
\email{p.barthe441@hotmail.com}                      
\homepage{paulbarza.github.io/Cv_Paul_Barthe/}                
\social[github]{PaulBarZa}                
\extrainfo{Nationalité Française\\Permis B\\58 Boulevard Desgranges, Sceaux 92330, France} 

\begin{document}

\maketitle

\smash{
    \begin{tikzpicture}
        \clip (0,0) circle (2cm) node {\includegraphics[width=4.5cm]{photo.jpg}};
    \end{tikzpicture}
}

\begingroup
\setlength{\parindent}{0.3cm}\textbf{\LARGE Paul BARTHE}
\endgroup

\medskip

\begingroup
\setlength{\parindent}{1cm}\textbf{\large À la recherche d'un projet de fin d'étude de 6 mois dans le domaine de l'intelligence artificielle.}
\endgroup

\smallskip

\section{FORMATIONS}
\cventry{2017--Maintenant}{Formation d'ingénieur en 5 ans, Spécialité informatique}{EPF école d'ingénieurs}{Sceaux, France}{}{}
\cventry{2014--2017}{Obtention du baccalauréat scientifique avec mention}{Lycée Salvador Allende}{Caen, France}{}{}

\section{EXPÉRIENCES}
\cventry{2020}{Stagiaire en développement web (6 mois)}{Actility}{Caen, France}{}{
    Développement en équipe d'une application web, Frontend et Backend, visant à piloter un réseau LoRaWAN.\\
    Apprentissage du fonctionnement d'un réseau LoRaWAN.\\
    Technologies utilisées : Angular et Go.
}

\section{PROJETS}
\cventry{Mars - Juin 2021}{IA Jeu du démineur}{}{}{}{
    Développement de deux modèles de machine learning jouant au jeu du démineur.
    L'un par satisfaction de contraintes et logique et le second par apprentissage profond par renforcement.\\
    Technologies utilisées : Python et Tensorflow
}
\cventry{Avril - Juin 2021}{Application Android}{}{}{}{
    Développement d'une application permettant de scanner des produits et d'obtenir des informations dessus.\\
    Technologies utilisées : Kotlin et Android Studio Code.
}
\cventry{2021}{Mon site web}{}{}{}{
    Développement d'un site web ayant pour objectif de me présenter.\\
    Technologies utilisées : HTML/CSS, JavaScript et NodeJs.
}

\section{COMPÉTENCES}
\subsection{À l'aise avec}
\cvline{Langages}{Python, Typescript, Golang, JavaScript, HTML, CSS, JSON}
\cvline{Technologies}{Angular, Echo, Tensorflow, Keras, Git, Visual Studio Code, Bootstrap, Ubuntu}
\subsection{Expérience avec}
\cvline{Langages}{PHP, C++, C Sharp, Java, Kotlin, SQL, Matlab, LateX}
\cvline{Technologies}{Symfony, Eclipse, MySQL, Android Studio, Unity}
\subsection{Langues}
\cvline{}{Français (langue native), Anglais niveau C1 (TOEIC 890/950) et Espagnol niveau B2.}
\subsection{Soft skills}
\cvline{}{Curieux, force de proposition et dynamique}

\section{ACTIVITÉS}
\cventry{2019}{Association d'Eloquence}{}{}{}{
    Responsable de la communication de l'association EPF Eloquence.
}
\cventry{2019}{Bureau des jeux}{}{}{}{
    Responsable du pôle poker de l'association du Bureau des Jeux de l'EPF.
}
\cventry{}{Centres d'intérêt}{}{}{}{
    Intelligence artificielle, Programmation, Course à pied, Natation, Cyclisme, Piano.
}

\end{document}
