%% Copyright 2006-2010 Xavier Danaux (xdanaux@gmail.com).
%% Copyright 2010-2011 Mark Liu (markwayneliu@gmail.com).
%
% This work may be distributed and/or modified under the
% conditions of the LaTeX Project Public License version 1.3c,
% available at http://www.latex-project.org/lppl/.

\documentclass[12pt,a4paper,sans]{moderncv}

\usepackage{verbatim}

% moderncv themes
\moderncvstyle{classic}
\moderncvcolor{blue}

% character encoding
\usepackage[utf8]{inputenc}                   

% adjust the page margins
\usepackage[left=.6in,right=.6in,top=.5in,bottom=.8in]{geometry}
\setlength{\hintscolumnwidth}{3cm}						

% personal data
\firstname{Paul}
\familyname{Barthe}
\mobile{07 81 09 14 85}                    
\email{p.barthe441@hotmail.com}                      
\homepage{https://paulbarza.github.io/Website/}                
\social[github]{PaulBarZa}                
\extrainfo{23 ans}                

\begin{document}
\maketitle

\begingroup
\setlength{\parindent}{2cm}\textbf{}
\endgroup

\section{Formations}
\smallskip
\cventry{2017--2022}{Formation d'ingénieur en 5 ans, Spécialité informatique}{EPF école d'ingénieurs}{Sceaux, France}{}{}
\cventry{2014--2017}{Obtention du baccalauréat scientifique avec mention}{Lycée Salvador Allende}{Caen, France}{}{}

\smallskip

\section{Expériences}
\smallskip
\smallskip
\cventry{Depuis août 2022}{Ingénieur R\&D}{Datexim}{Caen, France}{}{
    Formation et débuts de travaux sur des problématiques de recherche.\\
    Développement Backend et Frontend.\\
    Technologies utilisées: Python, Vue3 et Django.
}
\cventry{2022}{Projet de fin d'étude (6 mois)}{Cytomine}{Liège, Belgique}{}{
    Développement de solutions de machine learning intégrées dans la plateforme open source Cytomine, permettant l'analyse d'images médicales de très grandes tailles.\\
    Analyse et amélioration du système de lancement des algorithmes.\\
    Développement Frontend et Backend.\\
    Technologies utilisées: Python, Java (Spring Boot), VueJS et Docker.
}
\smallskip
\cventry{2021}{Développement front end}{Gaming Squad}{Paris, France}{}{
    Développement d'un client lourd pour l'entreprise Gaming Squad.\\
    Technologies utilisées: ReactJS.
}
\smallskip
\cventry{2020}{Stagiaire en développement web (6 mois)}{Actility}{Caen, France}{}{
    Développement Frontend et Backend d'une application de pilotage d'un réseau LoRaWAN.\\
    Technologies utilisées: Angular et Go.
}

\smallskip

\section{Projets}
\smallskip
\cventry{2021}{Classification et génération de sushi}{}{}{}{
    Développement d'un modèle de classification et génération de sushi en 50 types.\\
    Un réseau de neurones convolutif permettant l'extraction d'informations suivit de forêts d'arbres décisionnels pour le traitement de ces informations.\\
    Utilisation d'un réseau antagoniste génératif pour la génération.\\
    Technologies utilisées: Python, Tensorflow et Sklearn.
}
\smallskip
\cventry{2021}{IA Jeu du démineur}{}{}{}{
    Développement de deux modèles de machine learning jouant au jeu du démineur, le premier par satisfaction de contraintes et logique et le second par une méthode d'apprentissage profond par renforcement.\\
    Technologies utilisées: Python et Tensorflow
}
\smallskip
\cventry{2021}{Mon site web}{}{}{}{
    Développement d'un site web ayant pour objectif de me présenter.\\
    Technologies utilisées: HTML/CSS, JavaScript et NodeJs.
}

\section{Expériences techniques}
\smallskip
\subsection{À l'aise avec}
\cvline{Langages}{Python, Java, Typescript, JavaScript, HTML, CSS}
\cvline{Technologies}{Tensorflow, Keras, Sklearn, Matplotlib, Numpy, Angular, Docker}
\smallskip
\subsection{Expérience avec}
\cvline{Langages}{C++, C Sharp, Golang, Kotlin, SQL, PHP, Matlab, LateX}
\cvline{Technologies}{Airflow, Prefect, MySQL, Android Studio, Echo, React, Symfony, Unity}

\section{Cours principaux}
\smallskip
\subsection{EPF école d'ingénieurs, Sceaux}
\smallskip
\cvline{Science des données}{Big Data, Intelligence artificielle, Machine learning, Réseaux de neurones, Logique et probabilité, Statistiques.}
\cvline{Informatique}{Python, C++, Java, Réseaux et infrastructure, Développement logiciel avancé, Cryptographie, Introoduction à l'informatique quantique, Cybersecurité, Internet des objets }
\cvline{Autres}{Processus de calcul, Électromagnétisme, Physique, Thermodynamique, Mechanique, Anglais, Espagnol.}

\bigskip

\section{Certifications}
\smallskip
\cventry{2022}{Machine Learning - Standford University}{}{}{}{}
\smallskip
\cventry{2021}{TOEIC - 895/990}{}{}{}{}

\section{Activités}
\smallskip
\cventry{2019}{Association d'Eloquence}{}{}{}{
    Responsable de la communication de l'association EPF Eloquence.
}
\smallskip
\cventry{2019}{Bureau des jeux}{}{}{}{
    Responsable du pôle poker de l'association du Bureau des Jeux de l'EPF.
}
\smallskip
\cventry{}{Centres d'intérêt}{}{}{}{
    Intelligence artificielle, Machine learning, Programmation, Course à pied, Natation, Vélo, Piano, Jeu de Go.
}

\end{document}
